\documentclass[11pt, a4paper]{article}

% --- Packages ---
\usepackage[utf8]{inputenc}
\usepackage{geometry}
\geometry{top=2.5cm, bottom=2.5cm, left=2.5cm, right=2.5cm}
\usepackage{amsmath, amssymb, amsfonts}
\usepackage{graphicx}
\usepackage{hyperref}
\usepackage{booktabs}
\usepackage{fancyhdr}
\usepackage{float}
\usepackage{listings}
\usepackage{xcolor}
\usepackage[numbers]{natbib}

% --- Title & Author Data ---
\title{\textbf{The Theory of Topological Time-Stitching} \\ \large A Unified Geometric Framework for Discrete Spacetime, Quantum Branching, and Relativistic Mechanics}

% --- AUTHOR BLOCK ---
\author{
    \textbf{Jose L. S. Alvarenga} \\
    \textit{Independent Researcher} \\
    \textit{Indaiatuba, Brazil} \\
    \texttt{joselucas@gmail.com}
}
\date{November 30, 2025}

% --- Header/Footer Setup ---
\pagestyle{fancy}
\fancyhf{}
\rhead{Topological Time-Stitching}
\lhead{\today}
\cfoot{\thepage}

\begin{document}

\maketitle

\begin{abstract}
This paper proposes a shift from a ``Block Universe'' to a ``Process Universe,'' where time is implemented as a discrete, iterative operation on a stochastic causal network updated at the Planck scale. We introduce Topological Stitching, in which particles are modeled as propagating helical excitations defined by a discrete, iterative phase-locking cycle. Imposing a strict causal speed limit on the total vector velocity of the helix yields the de Broglie wavelength, the Lorentz factor of Special Relativity, and the Gullstrand–Painlevé form of the Schwarzschild metric from a single geometric principle. A topological closure condition further produces discrete stability bands, offering a geometric origin for quantum energy quantization and particle mass spectra. The framework suggests experimentally testable predictions, including drift quantization and gravitational flow-clock asymmetries.
\end{abstract}

\tableofcontents
\newpage

% ----------------------------------------------------------------------
\section{Introduction: The Process Universe}
% ----------------------------------------------------------------------

Standard physics typically models the universe as a ``Block'' (a pre-existing 4D spacetime manifold). This theory proposes a paradigm shift to a \textbf{``Process Universe,''} where Time is not a dimension but a discrete, iterative operation (a ``Tick'') occurring on a \textbf{Discrete Causal Network}.

The fundamental proposition is that particles are not continuous entities but \textbf{propagating helical excitations} defined by a discrete, iterative phase-locking cycle via a geometric mechanism termed \textbf{Topological Stitching}. This framework unifies the probabilistic nature of Quantum Mechanics with the deterministic geometry of General Relativity through a single generative principle: the conservation of causal continuity.

% ----------------------------------------------------------------------
\section{Conceptual Framework}
% ----------------------------------------------------------------------

\subsection{The Discontinuity Problem}
In a discrete universe, time is not a smooth flow but a series of ``frames.'' A particle existing in Frame 1 must calculate its position in Frame 2. We propose that particles act as \textbf{Helical Excitations} that must geometrically ``stitch'' themselves to the network nodes at each step.

\begin{figure}[H]
    \centering
    % Placeholder for the 2-Panel Schematic (Visual Abstract)
    % UNCOMMENT BELOW IF FILE EXISTS:
    \includegraphics[width=1.0\textwidth]{diagram.png}
    \caption{\textbf{Conceptual Overview.} (A) The physical intuition: Gravity acts as a river of space flowing into mass. A stationary particle must swim against this flow. (B) The geometric consequence: The ``Swim Speed'' creates a drift in the internal phase space, distorting the stitch cycle and slowing down proper time (Time Dilation).}
    \label{fig:concept}
\end{figure}

% ----------------------------------------------------------------------
\section{Foundations of the Model}
% ----------------------------------------------------------------------

\subsection{The Physical Substrate: The Stochastic Causal Network}
To preserve Lorentz Invariance, we define the substrate as a \textbf{Stochastic Causal Network} generated via Poisson Sprinkling \cite{bombelli1987}.
\begin{itemize}
    \item \textbf{Scale:} Average node separation $L_0$ (Planck Length).
    \item \textbf{Dynamics:} Universal update frequency $f_0$ (Planck Frequency).
    \item \textbf{Isotropy:} The Poisson distribution ensures no preferred reference frame exists at macroscopic scales.
    \item \textbf{Connectivity:} Mass is defined by \textbf{High Causal Valence} (local connectivity density).
\end{itemize}

\subsection{The Master Equation (Helical Geometry)}
Particles are modeled as \textbf{3D Helical Excitations} propagating through this network. The trajectory $\vec{S}(t)$ describes the state within the hidden phase space:

\begin{equation}
    \vec{S}(t) = \langle A \sin(\omega t), A \cos(\omega t), v_d t \rangle
\end{equation}

\noindent\textit{Clarification:} The helix represents an internal periodic degree of freedom associated with each particle. We do not postulate a specific physical medium that oscillates; rather, only the geometric relations of this phase degree of freedom influence observable dynamics.

\subsection{The Fundamental Constraint (Strict Mode)}
To ensure causality, the \textbf{Total Vector Velocity} cannot exceed the network update speed $c$:

\begin{equation}
    |\vec{v}_{total}|^2 = (A\omega)^2 + v_d^2 = c^2
\end{equation}

This implies the dynamic amplitude constraint:
\begin{equation}
    A = \frac{\sqrt{c^2 - v_d^2}}{\omega}
\end{equation}
For a particle at rest ($v_d=0$), this reduces to $A = c/\omega$, naturally deriving the \textbf{De Broglie Wavelength}.

% ----------------------------------------------------------------------
\section{Formal Gauge Derivation (Dynamics)}
% ----------------------------------------------------------------------

\subsection{The Connection 1-Form ($A_\mu$)}
We map the electromagnetic potential $A_\mu = (\phi, \vec{A})$ to the geometric parameters:
\begin{enumerate}
    \item \textbf{Scalar Potential:} $q\phi \equiv \hbar \frac{v_d}{A}$
    \item \textbf{Vector Potential:} $q\vec{A} \equiv \hbar \vec{k}_{rot}$
\end{enumerate}

\subsection{Emergent Dynamics (Microscopic Hamiltonian)}
The field dynamics emerge from the elasticity of the causal lattice. The energy cost to distort the phase alignment scales quadratically, recovering the Maxwell Action $S_{eff} \propto \int F^2$ in the continuum limit.

% ----------------------------------------------------------------------
\section{Quantum Mechanics: Quantization \& Stability}
% ----------------------------------------------------------------------

\subsection{The Zero-Drift Condition}
For a particle to persist, the ``Stitch'' must form a closed loop in phase space. The integrated vertical drop of the wave during the Backswing must exactly negate the integrated linear drift.

\textbf{Result: The Stability Equation}
\begin{equation}
    \boxed{ \sqrt{1 - \left(\frac{v_d}{c}\right)^2} = \frac{v_d}{c} \left( \pi N - \arccos\left(-\frac{v_d}{c}\right) \right) }
\end{equation}

\subsection{The Quantized Drift Spectrum}
Solving this yields discrete allowed drift velocities ($v_d$) for harmonic integers ($N$). See Appendix C for the full spectrum.

\subsection{The Continuum Limit (Schrödinger Derivation)}
By expanding the unitary update operator of the helical stitch $U(t, t+\delta t)$ acting on a slow envelope function $\Psi$, we recover the linear Schrödinger equation $i\hbar \partial_t \Psi = \hat{H} \Psi$. (See Appendix D).

% ----------------------------------------------------------------------
\section{Unification with Special Relativity}
% ----------------------------------------------------------------------

\subsection{Kinematic Dilation (The Helical Derivation)}
Numerical analysis confirms that the Helical Stitch Ratio $R$ scales perfectly with Einstein's Lorentz Factor $\gamma$. This is because the ``Stitch'' measures the arc length of the helix required to complete one cycle ($T \propto \text{Circumference} / v_{rot}$), while $v_{rot} = \sqrt{c^2 - v_d^2}$.

\begin{equation}
    \gamma = \frac{1}{\sqrt{1 - \frac{v_d^2}{c^2}}}
\end{equation}

\begin{figure}[H]
    \centering
    % UNCOMMENT BELOW IF FILE EXISTS:
    \includegraphics[width=0.9\textwidth]{helical_derivation.png}
    \caption{\textbf{Geometric Verification of Time Dilation.} The Green Line represents the dilation predicted by the Helical Stitching Model. The Blue Dashed Line is the standard Einstein Gamma factor. The perfect overlap confirms that Special Relativity is the natural geometric limit of the helical stitch.}
    \label{fig:relativity}
\end{figure}

% ----------------------------------------------------------------------
\section{General Relativity as Emergent Causal Flow}
% ----------------------------------------------------------------------

\subsection{Gravity as Causal Valence}
Mass corresponds to a region of \textbf{High Causal Valence} (increased connectivity). To maintain a uniform update rate $f_0$, the network connections must ``flow'' inward.
\begin{itemize}
    \item \textbf{Inward Flow Velocity:} $v_{\text{flow}}(r) = -c \sqrt{\frac{R_s}{r}}$.
\end{itemize}

\subsection{Time Dilation \& Metric}
A stationary particle ``swims'' against this flow ($v_d = -v_{\text{flow}}$). The internal rotation is slowed by the Helical Constraint, reproducing Schwarzschild time dilation. This flow reconstructs the \textbf{Gullstrand–Painlevé metric}.

\begin{figure}[H]
    \centering
    % UNCOMMENT BELOW IF FILE EXISTS:
    \includegraphics[width=1.0\textwidth]{gravity_river_viz.png}
    \caption{\textbf{The River Model of Gravity.} (A) The physical frame: Space flows inward towards the mass (Blue Arrow). A stationary particle must swim outward (Green Arrow) to maintain its position. (B) The Phase Space consequence: The ``Swim Speed'' acts as the Drift parameter $m$ ($v_d$), stretching the Manifest Phase (Blue) and compressing the Backswing (Orange), causing Time Dilation.}
    \label{fig:river}
\end{figure}

% ----------------------------------------------------------------------
\section{Experimental Proposals}
% ----------------------------------------------------------------------

\subsection{Proposal A: Drift Quantization Interferometry}
\textbf{Hypothesis:} Velocity is stepped at small $N$.
\begin{itemize}
    \item \textbf{Estimate:} For ${}^{87}Rb$ atoms ($N \approx 10^{10}$), $\Delta v \approx 10^{-12}$ m/s.
    \item \textbf{Feasibility:} Detectable via Large Momentum Transfer (LMT) atom interferometry.
\end{itemize}

\subsection{Proposal B: The Flow-Clock Experiment}
\textbf{Test:} Compare atomic clocks moving with vs. against gravitational frame-dragging.

% ----------------------------------------------------------------------
\section{Fundamental Constants \& Symbol Table}
% ----------------------------------------------------------------------

\subsection{Constants}
\begin{enumerate}
    \item \textbf{Speed of Light ($c$):} Max vector update speed.
    \item \textbf{Planck's Constant ($h$):} Minimal Loop Area in Phase Space.
    \item \textbf{Fundamental Energy ($E_0$):} $E_0 = h \cdot f_0$ (Grid Capacity).
\end{enumerate}

\subsection{Relativistic Energy Balance}
The total energy of the excitation scales with the Time Dilation factor (Stitch Ratio) $\gamma$:
\begin{equation}
    E_{total} = \hbar \omega_{rest} \cdot \gamma = \frac{\hbar \omega_{rest}}{\sqrt{1 - \frac{v_d^2}{c^2}}}
\end{equation}
The Kinetic Energy is the excess energy required to maintain the stitch against the drift:
\begin{equation}
    K = E_{total} - E_{rest} = \hbar \omega_{rest} (\gamma - 1)
\end{equation}

\subsection{Symbol Table}
\begin{table}[h]
\centering
\begin{tabular}{@{}llcl@{}}
\toprule
\textbf{Symbol} & \textbf{Definition} & \textbf{Units} & \textbf{Role} \\ \midrule
$f_0$ & Fundamental Frequency & $s^{-1}$ & Lattice update rate \\
$v_d$ & Stitch Drift & $m/s$ & Relative velocity \\
$\omega$ & Particle Frequency & $rad/s$ & Internal energy clock \\
$A$ & Amplitude & $m$ & Spatial size of internal helix \\
$N$ & Harmonic Number & Integer & Quantization band index \\
$v_{flow}$ & Grid Flux Velocity & $m/s$ & Speed of space flow (Gravity) \\ 
$h$ & Planck's Constant & $J \cdot s$ & Minimal Loop Area \\ \bottomrule
\end{tabular}
\caption{Glossary of Symbols and Physical Constants}
\end{table}

% ----------------------------------------------------------------------
\section{Conclusion}
% ----------------------------------------------------------------------
The \textbf{Topological Time-Stitching Theory} suggests that the ``weirdness'' of quantum mechanics and the ``warping'' of relativity are not separate phenomena. They are simply the low-speed and high-speed behaviors of the same underlying mechanism: \textbf{A discrete, resonant processing cycle that creates reality one stitch at a time.}

% ----------------------------------------------------------------------
\section{Data and Code Availability}
% ----------------------------------------------------------------------
The numerical verifications and visualizations presented in this paper were generated using open-source Python simulations. The source code is available to allow reviewers to reproduce the derivations.

\begin{enumerate}
    \item \textbf{Figure 2:} \texttt{helical\_relativity.py} - Verifies the geometric derivation of the Lorentz Factor $\gamma$.
    \item \textbf{Figure 3:} \texttt{gravity\_river.py} - Visualizes the grid flux field near the Event Horizon.
    \item \textbf{Figure 4:} \texttt{quantized\_spectrum.py} - Solves the transcendental stability equation for $N=1..20$.
\end{enumerate}

Repository URL: \url{https://github.com/joselucasbr/topological-stitching} \textit{(Placeholder)}

% ----------------------------------------------------------------------
% Bibliography
% ----------------------------------------------------------------------
\begin{thebibliography}{9}

\bibitem{bombelli1987}
L. Bombelli, J. Lee, D. Meyer, and R. D. Sorkin, 
``Space-time as a causal set,'' 
\textit{Phys. Rev. Lett.}, vol. 59, p. 521, 1987.

\bibitem{sorkin2003}
R. D. Sorkin, 
``Causal Sets: Discrete Gravity,'' 
\textit{Lectures on Quantum Gravity}, Plenum Press, 2005.

\bibitem{rovelli1998}
C. Rovelli, 
``Loop Quantum Gravity,'' 
\textit{Living Rev. Relativ.}, vol. 1, 1998.

\bibitem{padmanabhan2010}
T. Padmanabhan, 
``Thermodynamical Aspects of Gravity: New insights,'' 
\textit{Rep. Prog. Phys.}, vol. 73, 2010.

\bibitem{verlinde2011}
E. Verlinde, 
``On the Origin of Gravity and the Laws of Newton,'' 
\textit{JHEP}, vol. 2011, 2011.

\bibitem{painleve1921}
P. Painlevé, 
``La mécanique classique et la théorie de la relativité,'' 
\textit{C. R. Acad. Sci.}, vol. 173, 1921.

\bibitem{gullstrand1922}
A. Gullstrand, 
``Allgemeine Lösung des statischen Einkörperproblems in der Einsteinschen Gravitationstheorie,'' 
\textit{Ark. Mat. Astron. Fys.}, vol. 16, 1922.

\end{thebibliography}

% ----------------------------------------------------------------------
% Appendices
% ----------------------------------------------------------------------
\appendix
\newpage

\section{Derivation of the Stability Condition}
\begin{enumerate}
    \item \textbf{Loop Closure:} The vertical velocity of the wave is $v_y(t) = v_{rot} \cos(\omega t) + v_d$.
    \item \textbf{Critical Angle:} The ``Backswing'' begins when $v_y < 0$. This occurs at $\theta_{crit} = \arccos(-v_d/v_{rot})$.
    \item \textbf{Strict Mode:} At limit $v_{total}=c$, $v_{rot} \approx c$. Thus $\theta_{crit} = \arccos(-v_d/c)$.
    \item \textbf{Integration:} Integrating $v_y(t)$ over one period yields the transcendental equation. \hfill \textit{(Q.E.D.)}
\end{enumerate}

\section{Microscopic Action}
The lattice Hamiltonian $H = \sum \kappa (\theta_i - \theta_j - A_{ij})^2$ minimizes phase distortion. In the continuum limit, $(\nabla \theta - A)^2 \to F_{\mu\nu}^2$, recovering the Maxwell Action.

\section{The Quantized Drift Spectrum}
This appendix details the numerical solutions to the Zero-Drift Stability Equation.

\begin{figure}[H]
    \centering
    % UNCOMMENT BELOW IF FILE EXISTS:
    \includegraphics[width=1.0\textwidth]{drift_spectrum.png}
    \caption{\textbf{Quantized Drift Spectrum.} Left: Allowed drift velocities decay as $1/N$. Right: The step size shrinks as $1/N^2$, recovering classical continuity for macroscopic objects.}
    \label{fig:spectrum}
\end{figure}

\section{Multiscale Schrödinger Derivation}
We model the lattice update as a unitary operator $U(t, t+\delta t)$ acting on the wavefunction $\Psi(x,t)$:
\begin{equation}
    \Psi(x, t+\delta t) = \sum_{y} K(x,y) \Psi(y,t)
\end{equation}
\begin{itemize}
    \item \textbf{Kernel $K(x,y)$:} A Gaussian distribution representing the ``Fraying'' of the helix over nearest neighbors.
    \item \textbf{Diffusion Constant:} $D \approx f_0 L_0^2$.
    \item \textbf{Expansion:} Expanding $\Psi$ to second order in space and first order in time recovers the diffusion equation $\partial_t \Psi = i D \nabla^2 \Psi$, which corresponds to the kinetic term of the Schrödinger equation.
\end{itemize}

\end{document}